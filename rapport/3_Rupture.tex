\chapter{Détection de rupture}

\section{Introduction}

Cette partie présentera les deux méthodes de détection de rupture implémentées : une méthode par dérivée filtrée utilisant la $p$-valeur décrite par \cite{Bertrand11} et \cite{Herault14}, et une méthode de détection de changement de noyau décrite par \cite{Desobry05}.

Cette détection de rupture sera appliquée sur une matrice de signal $X$ contenant $n$ lignes d'« observations », une par image de la vidéo, et $p$ colonnes de caractéristiques, 1 pour chaque caractéristique extraite de la vidéo.

\section{Dérivée filtrée}

\subsection{Principe}

Le principe de cette première méthode est de parcourir le signal temporel $X$ avec deux fenêtres de taille $A_1$ et $A_2$ respectivement avant et après une position $k$ (de $k - A_1$ à $k-1$ et de $k$ à $k+A_2-1$).

On calcule ensuite, pour chaque position $k$, une valeur de distance $D(k)$ entre des indicateurs $\hat{\theta}$ calculés sur chaque fenêtre.

On détermine ensuite les zones de rupture comme étant les zones telles que $|D(k)| > C$, avec $C$ un seuil fixé, puis on détermine pour chaque zone de rupture $\mathcal{K}_i$ la position $k_{r_i}$ exacte de la rupture comme étant $k_{r_i} = \displaystyle \argmax_{k \in \mathcal{K}_i} |D(k)|$.

Afin de déterminer le seuil $C$, il est possible d'utiliser une méthode probabiliste utilisant la $p$-valeur. En posant $M = \displaystyle \max_{k} |D(k)|$, et l'hypothèse $\mathcal{H}_0$ où il n'y a pas de rupture, on veut $C$ tel que $\mathbb{P}(M>C\mid	\mathcal{H}_0) < p$ avec $p$ fixé.

Pour estimer $C$, on calcule des estimations de réalisation de $M$ sous l'hypothèse $\mathcal{H}_0$ en permutant le signal temporel $X$ (cela force en quelque sorte $\mathcal{H}_0$) et en calculant $\hat{M}$ sur ce nouveau signal. Ces réalisations $\hat{M}$ permettent d'estimer la distribution de $M$. En supposant qu'il s'agisse d'une loi normale, on peut estimer ses paramètres $\hat{\mu}$ et $\hat{\sigma}$, et on peut ainsi estimer le seuil $C$ en fonction d'une valeur fixée de $p$.

\subsection{Résultats}

TODO

\section{Changement de noyau}

\subsection{Principe}

Le principe de cette méthode est identique au principe de base de la méthode de dérivée filtrée, on calcule une distance $D(k)$ entre deux fenêtres autour de $k$. Cependant, dans le cas de cette méthode, la distance $D(k)$ est calculée comme étant une distance entre les paramètres $\rho_i$ et $\alpha_i$ de deux \textit{one-class SVM} calculés respectivement sur chaque fenêtre.

Dans ce cas, pour chaque valeur de $k$, la distance est la suivante :

\begin{align}
  D &= \frac{\overset{\frown}{c_1 c_2}}{\overset{\frown}{c_1 p_1} + \overset{\frown}{c_2 p_2}} \notag \\
  \overset{\frown}{c_1 c_2} &= \arccos\left(\frac{
  	\alpha_1^\top K_{12} \alpha_2
  	}{
  	\sqrt{\alpha_1^\top K_{11} \alpha_1} \sqrt{\alpha_2^\top K_{22} \alpha_2}
  	} \right) \notag \\
  \overset{\frown}{c_i p_i} &= \arccos\left( \frac{\rho_i}{\sqrt{\alpha_i^\top K_{ii} \alpha_i}}  \right) \notag
\end{align}

avec $i$ et $j$ le numéro du \textit{one-class SVM}, $\alpha_i$ les variables duales, $\rho_i$ le biais et $K_{ij}$ la matrice de Gram avec les données $i$ et $j$.

\subsection{Résultats}

TODO

\nocite{Loosli05}
