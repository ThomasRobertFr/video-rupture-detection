\section{Récapitulatif sur les caractéristiques}

Comme il a été dit auparavant, chacune des caractéristiques extraites des images ont leurs avantages et leurs inconvénients. En fonction du type de rupture à détecter, il sera préférable d'utiliser l'une ou l'autre des caractéristiques, bien que certaines des caractéristiques présentées ici sont très proches et détecte le même type de rupture. Cependant, dans un cas plus général, on ne connaît pas \textit{a priori} les types de rupture qui composent une vidéo. Il conviendrait donc de combiner ces différentes caractéristiques, afin d'avoir un spectre de détection plus grand. Toutefois, il faut prendre en compte les faux-positifs. En effet, certaines des caractéristiques ont tendance à détecter des ruptures là où il n'y en a pas.\\

Le choix des caractéristiques est donc important pour la détection de rupture dans une vidéo.